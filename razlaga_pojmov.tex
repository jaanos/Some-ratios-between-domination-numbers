  \documentclass[12pt,a4paper]{amsart}
% ukazi za delo s slovenscino -- izberi kodiranje, ki ti ustreza
\usepackage[slovene]{babel}
%\usepackage[cp1250]{inputenc}
%\usepackage[T1]{fontenc}
\usepackage[utf8]{inputenc}
\usepackage{amsmath,amssymb,amsfonts}
\usepackage{url}
%\usepackage[normalem]{ulem}
\usepackage[dvipsnames,usenames]{color}


\textwidth 15cm
\textheight 24cm
\oddsidemargin.5cm
\evensidemargin.5cm
\topmargin-5mm
\addtolength{\footskip}{10pt}
\pagestyle{plain}
\overfullrule=15pt

\begin{document}

\section{Razlaga pojmov}

\textit{Graf G} je množica točk v prostoru in povezav med temi točkami. Označimo ga z $G=(V, E)$, kjer je $V(G)$ množica točk in $E(G)$ množica povezav grafa $G$.
\textit{Odprta okolica ali soseščina N} vozlišča $v$ je množica vozlišč, ki je sosedna vozlišču $v$, torej $N(v) =\{ u\in V: uv\in E \}$. \textit{Kartezični produkt grafov} $G_1 = (V_1, E_1)$ in $G_2 = (V_2, E_2)$ je graf $G = G_1 \square G_2$, ki ima množico točk $ V(G) = V_1 \times V_2$ in množico povezav $ E(G)$, kjer je $(u,v)(x,y) \in E(G)$, če je $ u=x $ in $vy \in E(G_2) $ ali $ux \in E(G_1)$ in $v = y$.
\textit{Dominantna množica $ D \subseteq V(G) $} grafa $G$ je takšna množica, da ima vsako vozlišče grafa, ki ni v $D$ $( v \in V(G) \setminus D)$, soseda v $D$. Z drugimi besedami, vsako vozlišče  $v \in V(G)$ je ali  element množice $D$ ali pa je sosednje kakemu vozlišču, ki pripada množici $D$.  \\ \textit{ Dominantno število $\gamma(G) $} je moč najmanjše dominantne množice grafa $G$. \\
Množica $S$ je \textit{totalno dominantna}, če je $ N(S) = V(G)$, kar pomeni, da je vsako vozlišče iz $ V(G)$ sosednje vozlišču iz množice $S$. \\
 Z \textit{$ \gamma_t $ }označujemo \textit{totalno dominantno število}, ki predstavlja velikost najmanjše totalno dominantne množice. \\
 Podmnožica $S \subset V(G)$ je $\gamma_t$-set, če je to totalno dominantna množica grafa G, z močjo $\gamma_t(G)$.\\
%Množica $S \subseteq V(G) $ je\textit{ k-dominantna množica}, če za vsako vozlišče  $v \in V(G)$ velja, da je  $v \in S$ ali obstaja  $u \in S$, ki je na razdalji največ k od vozlišča v. Z $ \gamma_k (G) $ bomo označevali k-dominantno število grafa G, ki predstavlja moč najmanjše k-dominantne množice v G.\\
Naj bo $G=(V, E)$ graf in $f: V \to P (\{1,2\dots,k\})$ funkcija, ki vsakemu vozlišču iz $V$ priredi množico barv iz $ \{1,2\dots,k\}$. Če za vsak $v \in V$ za katerega je $ f(v) = \emptyset $ velja $ \bigcup_{u \in N(v) } f(u) = \{1,2\dots,k\}$ potem f imenujemo\textit{  k-mavrična dominantna funkcija } grafa $G$, krajše kRDF funkcija. \textit{Težo $\omega(f)$} funkcije f, definiramo z $\omega(f) = \sum_{v \in V} \mid f(v)\mid$. Najmanjša vrednost mavrične dominantne funkcije grafa $G$ se imenuje \textit{k-mavrično dominanto število}, in jo označimo z $\gamma_{rk} (g)$.\\
Za graf $G$ je \textit{k-mavrično totalno dominantna funkcija} f, krajše kRTDF,  k-mavrična dominantna funkcija s pogojem, da podgraf grafa G, ki ga določa množica $\{v \in V(G) \mid f(v) \neq \emptyset \}$ nima izoliranih vozlišč. Teža funkcije kRTDF je  $\omega(f) = \sum_{v \in V} \mid f(v)\mid$. Za dan graf G, imenujemo težo najmanjše kRTDF funkcije \textit{k-mavrično totalno dominantno število}, in jo označimo z  $\gamma_{rkt} (G)$.

\section{Dani problem}

1.
Najdi funkcijo $b(k)$, da za $k \geqq 3$, za katero je dana neenakost ozka:
\begin{equation}
  b(k) \cdot \gamma_t \geqq \gamma_{krt}(G).
\end{equation}
Z drugimi besedami, najdi $b(k) = \inf_G \frac{\gamma_{krt}(G)}{\gamma_t(G)}$.

2.
Najdi funkcijo $a(k)$, da za $k \geqq 3$, za katero je dana neenakost ozka:
\begin{equation}
  \gamma_{krt}(G) \geqq a(k) \cdot \gamma_{kr}(G).
\end{equation}
Z drugimi besedami, najdi $a(k) = \sup_G \frac{\gamma_{krt}(G)}{\gamma_{kr}(G)}$

\pagebreak

\section{Neenakosti, dokazane v pdfju}
Total k-Rainbow domination numbers in graphs

\subsection{Izrek 2}
Naj bo $k \geqq 2, k\in \mathbb N$ in $G$ povezani graf stopnje $n \leqq k$.
Potem:
\begin{equation}
  \gamma_t (G) \leqq \gamma_{trk}(G) \leqq k\cdot \gamma_t(G).
\end{equation}
 Zanima nas le zgornja meja neenakosti, saj je to rešitev prvega zastavljenega problema.
\proof
Naj bo $S$ $\gamma_t(G)$ set in definiramo $g: V(G) \rightarrow 2^{[k]}$, kjer $g(x) = \{1,2,\ldots,k\}$ za $x \in S$ in $g(x) = \emptyset$ za $x \in V(G)\backslash S$. Torej je $g$ kRTFD za G in $\gamma_{trk}(G) \leqq \omega(g) = k\cdot \gamma_t(G)$
\endproof
Enakost je dosežena natanko tedaj, ko ima $G$ $\gamma_{trk}(G)$ funkcijo, $f$, da je za $\forall v \in V(G)$ ali $f(v) = \{1,2,\ldots,k\}$ ali $f(v) = \emptyset$.


\end{document}
